\chapter{Литературный обзор} \label{ch:ch1}

%------------------------------------------------------------------------------------------
\section{Удаление меркаптанов} \label{sec:ch1/sec1}

Меркаптаны, или, согласно более точному термину, тиолы, представляют собой класс органических соединений, содержащих группу $SH$. Их структурная формула обычно записывается как $RSH$, где $R$ может представлять собой алкильную или ароматическую группу. Меркаптановые компоненты обычно содержаться в природном газе и жидком топливе, таких как бензин, керосин, дизельное топливо. Необходимость удаления меркаптанов обусловлена несколькими факторами:
	
\begin{enumerate}
	\item Их кислые свойства могут привести к серьезным проблемам с коррозией;
	\item Неприятный запах, обусловленный наличием меркаптанов, требует их удаления из топлива перед сжиганием;
	\item Большинство меркаптанов обладают высокой токсичностью.
\end{enumerate}

Меркаптаны преимущественно представлены низкомолекулярными $C _1$ – $C _3$ прямо-цепными соединениями в составе ПГ и СУГа, в то время как в более тяжелых фракциях присутствуют разветвленные и более высокомолекулярные меркаптаны. Исследования по обработке топлива, содержащего меркаптаны, начались в 1860 году, и традиционные методы удаления основаны на использовании неорганических солей, первый метод «метод докторской очистки» описан в патенте \cite{kalinowski_doctor_1959}. и второй метод так называемая «медная очистка» \cite{krause_color_1952}. В этих методах меркаптаны не удаляются из топлива, а превращаются в дисульфиды, которые не проявляют коррозионную активность и обладают относительно незначительным запахом. Следовательно, содержание серы в топливе не снижается в результате самой очистки обоими упомянутыми традиционными методами.

Метод докторской очистки, который является наиболее старым из них, основан на использовании соли свинца: плумбита натрия ($Na_2PbO_2$). На первом этапе раствор $Na_2PbO_2$, полученный путем растворения оксида свинца в растворе гидроксида натрия ($NaOH$), смешивается с топливом, подлежащим обработке. Меркаптаны в топливе реагируют с $Na_2PbO_2$, образуя свинцовый меркаптид ($(RS)_2Pb$), который растворяется в топливе по реакции \cref{eq:reaction1}:

\begin{equation}
	\mathrm{Na_2PbO_2 + 2RSH \rightarrow {(RS)_2Pb} + 2NaOH} \label{eq:reaction1}
\end{equation}

$(RS)_2Pb$ может быть вновь превращен в $Na_2PbO_2$ с использованием воздуха по реакции \cref{eq:reaction2}: 

\begin{equation}
	\mathrm{(RS)_2Pb + 2NaOH + 1/2O_2 \rightarrow RSSR + Na_2PbO_2 + H_2O} \label{eq:reaction2} 
\end{equation} 

Однако восстановление раствора $Na_2PbO_2$ этим методом происходит медленно и лишь частично. Для полного превращения $(RS)_2Pb$ в дисульфид $(RSSR)$ в реакционную смесь добавляется молекулярная сера, и $(RS)_2Pb$ полностью превращается в свинцовый сульфид $(PbS)$, который выпадает из раствора по реакции \cref{eq:reaction3}:

\begin{equation}
	\mathrm{(RS)_2Pb + S \rightarrow RSSR + PbS} \label{eq:reaction3}
\end{equation} 

Этот процесс обеспечивает полное превращение, однако свинцовый сульфид не может быть восстановлен, поэтому его необходимо утилизировать при завершении реакции. Более того, поскольку сера, добавленная в реакционную смесь для образования дисульфидов, сама растворяется в топливе, содержание серы увеличивается после обработки. В связи с этим данный процесс редко применяется сегодня, за исключением случаев аналитического использования, включенный в стандарты ASTM (ASTM D 4952-09) \cite{noauthor_doctor_nodate}.

Еще один традиционный метод превращения меркаптанов, который называется «медной очисткой» в котором медь используется в виде хлорида ($CuCl_2$). На первом этапе меркаптаны окисляются до дисульфидов по реакции \cref{eq:reaction4}:

\begin{equation}	
	\mathrm{2RSH + 2CuCl_2 \rightarrow RSSR + 2CuCl + 2HCl} \label{eq:reaction4}
\end{equation}

Затем суспензию хлорида меди $(I)$ окисляют снова до хлорида меди $(II)$ с помощью воздуха, в соответствии с реакцией \cref{eq:reaction5}:
\begin{equation}	
	\mathrm{2CuCl + 2HCl + 1/2O_2 \rightarrow 2CuCl_2 + 2H_2O} \label{eq:reaction5}
\end{equation}

Хотя эти процессы все еще используются в некоторых областях, технологии, которые описываются далее, обычно предпочтительнее с точки зрения производительности и затрат.

Существует два основных способа обработки больших объемов меркаптанов в углеводородном сырье: экстракция, где легкие меркаптаны окисляются и удаляются в виде дисульфидов, и очищение, где тяжелые меркаптаны также окисляются, но остаются в потоке. Эти методы применяются при удалении \num{200} кг серы эквивалентов и более в день. Для потоков с меньшим количеством меркаптанов (менее \num{200} кг серы эквивалентов в день) используются альтернативные методы, такие как промывка.

Два основных поставщика лицензий на процессы удаления больших количеств меркаптанов - UOP (Merox\cite{farshi_kinetic_2005}) и GTP-Merichem (Merichem\cite{kohl_gas_1997}). В зависимости от характеристик необходимой степени очистки сырья, каждая компания использует различные конфигурации процессов \cite{bricker_advances_2012, noauthor_mericat_nodate, noauthor_thiolex_nodate}, что отображено в Таблице \cref{tab:remove}.

\begin{table}
	\centering
	\fontsize{12}{15}\selectfont % Установка размера шрифта
	\captionsetup{justification=centering} % выравнивание подписи по-центру
	\caption{Процессы удаления меркаптанов с большим их содержанием \cite{de_angelis_natural_2012}.}\label{tab:remove}
	\begin{tabular}{lllllc}
		\toprule
		Сырье   & Макс. содержание & Процесс      & Процесс    & Содержание после &  \\
		        & до обработки     & Merox        & Merichem   & обработки        &  \\
		        & (ppm Серы)       &              &            &                  &  \\ \midrule
		ПГ      & \num{10000}      & Merox        & Merichem   & Сокращение       &  \\
		СУГ     &                  & Абсорбция,   & Абсорбция, &                  &  \\
		        &                  & Экстракция   & Экстракция &                  &  \\
		Нафта   & \num{5000}       & Экстракция   & Экстракция & Неизменная       &  \\
		бензин  &                  & fixed bed    &            &                  &  \\
		        &                  & sweetening,  &            &                  &  \\
		        &                  & Minalk       &            &                  &  \\
		        &                  & sweetening   &            &                  &  \\
		Нафта   & \num{1000}       & fixed bed    & fixed bed  & Неизменная       &  \\
		Керосин &                  & sweetening,  & sweetening &                  &  \\
		ДТ      &                  & caustic free &            &                  &  \\
		        &                  & sweetening   &            &                  &  \\ \bottomrule
	\end{tabular}
\end{table}

Перед каждым применением процессов Merox или Merichem необходима предварительная щелочная очистка с разведенным раствором гидроксида натрия ($NaOH$), чтобы удалить весь $H_2S$ в сырье. При этом содержание $H_2S$ не должно превышать \num{10} ppm, так как в противном случае $H_2S$ может необратимо реагировать с сильными щелочными растворами, используемыми в процессах Merox или Merichem. Процессы Merox имеют более широкое распространение и масштабы, превышающие \num{1,8} млн. тонн в сутки. К январю \num{2002} года введено в эксплуатацию почти \num{1600} установок с производительностью от \num{5,5} до \num{19,096} тонн в сутки. Для обработки тяжелого сырья, такого как нефть, керосин, дизельное топливо и отопительные масла, также необходима предварительная щелочная очистка для удаления нафтеновых кислот.

Для обработки легкого углеводородного сырья, таких как ПГ или  СУГ, содержащие в себе низкомолекулярные меркаптаны (от метилмеркаптана до бутилмеркаптана), применяются процессы абсорбции и экстракции. На первом этапе газовый поток взаимодействует с концентрированным раствором гидроксида натрия (10-20\% массовой концентрации $NaOH$), в котором меркаптаны растворяются в виде соответствующих натриевых солей по реакции \cref{eq:reaction6}:

\begin{equation}
	\label{eq:reaction6}
	\mathrm{RSH_\text{(газофазная или жидкофазная)} + NaOH \rightarrow RSNa_\text{(жидкофазная)}} 
\end{equation}

На первом этапе процесса, который осуществляется в колонном аппарате, меркаптаны взаимодействуют с щелочным раствором противотоком, в результате чего ПГ или СУГ проходит через верхнюю часть колонны. Вода, содержащая меркаптаны в виде натриевых солей, а также катализатор Merox или Merichem (комплекс фталоцианина кобальта), вытекает из нижней части колонны, затем подогревается и направляется на окисление в регенератор. В регенераторе щелочной раствор реагирует с воздухом, что приводит к окислению меркаптидов до дисульфидов по реакции \cref{eq:reaction7}:

\begin{equation}	
	\mathrm{4RSNa + O_2 + 2H_2O \rightarrow 2RSSR + 4NaOH} \label{eq:reaction7}
\end{equation}

Дисульфиды формируют новую маслянистую фазу, которая тяжелее воды, и эта двухфазная смесь направляется в сепаратор. После этого регенерированный щелочной раствор возвращается в колонну разделения, а маслянистая фаза направляется на соответствующие нужды, например, использование дисульфидного масла как сульфидирущего агента для катализатора гидроочистки, ингибитор коксообразования УЗК и при возможности, продажа как химического реагента. При использовании данных процессов периодически добавляется небольшое количество катализатора для поддержания достаточной эффективности процесса. При обработке методом абсорбции или экстракции меркаптаны превращаются в менее агрессивные и токсичные дисульфиды, а так же и содержание серы в газе уменьшается, поскольку образовавшиеся при окислении меркаптанов дисульфиды удаляются. Прямые операционные расходы оцениваются в \num{4,6} рублей за \num{1} кубический метр очищенного ПГ и \num{230} рублей за \num{1} кубический метр для очистки СУГ.

В таблицах \cref{tab:solid, tab:liquid} приведены наиболее распространенные хемосорбционные процессы для очистки углеводородного сырья в которых содержится малое количество сернистых соединений \cite{foral_evaluation_1995} (обычно менее \num{200} ppm, в пересчете на общую серу) с использованием твердых и жидких поглотителей. Большинство процессов представляют собой стехиометрические реакции между молекулами, содержащими серу, и реагентом, который не может быть восстановлен и должен быть утилизирован. Обычная аминовая очистка (МЕА, ДЕА и МДЕА) не являются экономически выгодными. Эти процессы предназначены для использования, где ежедневное производство соединений серы не превышает \num{185} кг серы в день.  

\begin{table}
	\centering
	\fontsize{12}{15}\selectfont % Установка размера шрифта
	\captionsetup{justification=centering} % выравнивание подписи по-центру
	\caption{Процессы для небольшого количества $H_2S$ и меркаптанов (твердая фаза)\cite{abdulrahman_natural_2012, kohl_gas_1997, de_angelis_natural_2012}.} \label{tab:solid}
	\begin{tabular}{lllllc}
		\toprule
		Название процесса & Активная        & Вид           & Возможность & Максимальная &  \\
		(компания)        & фаза            & процесса      & рецикла     & произв.,     &  \\
		                  &                 &               &             & кг/день      &  \\ \midrule
		Iron sponge       & Оксид железа на & Периодический & нет         & \num{45}     &  \\
		(Connelly-GPM)    & опилках дерева  &               &             &              &  \\
		Sulfatreat        & Гематит         & Периодический & нет         & \num{90}     &  \\
		(Sulfatreat)      &                 &               &             &              &  \\
		Sulphur-Rite      & Оксид железа    & Fixed bed     & нет         & \num{180}    &  \\
		(Merichem)        &                 &               &             &              &  \\
		Low temperature   & Оксид цинка     & Fixed bed     & нет         & \num{135}    &  \\
		zinc oxide (ICI)  &                 &               &             &              &  \\
		Chemsweet         & Оксид цинка +   & Периодический & нет         & \num{18}     &  \\
		(NATCO)           & ацетат цинка    &               &             &              &  \\
		Sulfosorb         & Активный уголь  & Fixed bed     & да          & \num{1,4}    &  \\
		(Calgon           & насыщенный      &               &             &              &  \\
		carbon co.)       & солями меди     &               &             &              &  \\ \bottomrule
	\end{tabular}
\end{table}

\begin{table}
	\centering
	\fontsize{12}{15}\selectfont % Установка размера шрифта
	\captionsetup{justification=centering} % выравнивание подписи по-центру
	\caption{Процессы для небольшого количества $H_2S$ и меркаптанов (жидкая фаза) \cite{de_angelis_natural_2012}.} \label{tab:liquid}
	\begin{tabular}{lllllc}
		\toprule
		Название процесса & Активная            & Вид           & Возможность & Максимальная &  \\
		(компания)        & фаза                & процесса      & рецикла     & произв.,     &  \\
		                  &                     &               &             & кг/день      &  \\ \midrule
		Sulfa-check       & Водный раствор      & Периодический & нет         & \num{90}     &  \\
		(Nalco-Exxon)     & $NaNO_2$            &               &             &              &  \\
		Sulfascrub        & 50\% водный раствор & Периодический & нет         & \num{45}     &  \\
		(Petrolite Co.)   & триазина            &               &             &              &  \\
		The Eliminator    & Раствор триазина    & Периодический & нет         & \num{90}     &  \\
		(Merichem)        &                     &               &             &              &  \\ \bottomrule
	\end{tabular}
\end{table}

%------------------------------------------------------------------------------------------
\subsection{Аппаратурное оформление процесса демеркаптанизации.} \label{sec:ch1/sec2}

Процесс удаления меркаптанов из СУГ может быть реализован различными способами, и выбор конкретного метода зависит от требований качества очистки, объемов обрабатываемого газа, экономической целесообразности и других факторов. На рисунке \cref{fig:ExtPlus} изображена схема экстракции процесса Merox "Extractor Plus" \cite{bricker_advances_2012}.

\begin{figure}
	\centering
	\includegraphics[width=1\linewidth]{images/Extplus}
	\caption{Схема демеркаптанизации компании Merox (Extractor Plus).}
	\label{fig:ExtPlus}
\end{figure}

Это стандартное предложение UOP, где удаление проводят в противоточном тарельчатом либо насадочном экстракторе, данная схема обеспечивает эффективное удаление меркаптановой серы из СУГ. На первом этапе в кубе колонны проводят предварительную очистку сырья с наибольшим содержанием меркаптанов с использованием свежей щелочи для удаления остаточного $H_2S$. Затем, поднимаясь вверх по колонне, сырье которое уже содержит в себе меньше меркаптанов контактирует с регенерированным щелочным раствором. Для обеспечения необходимого уровня удаления меркаптанов рассчитывают необходимый слой насадки или количество тарелок. Противоток в экстракторе обеспечивает благоприятное условие равновесия реакции \cref{eq:reaction6}. В верхней части колонны предусмотрен каплеотбойник для предотвращения уноса щелочи, а из куба колонны насыщенная меркаптидами щелочной раствор поступает на регенерацию. 

Регенерация щелочи осуществляется в отдельной колонне по реакции \cref{eq:reaction7}. Данную окислительную реакцию проводят с избытком воздуха в присутствии катализатора (Merox WSTM), реакция протекает только в сторону образования продукта и более высокая температура до \num{70}°C способствует увеличивает скорость реакции. Merox WSTM - это гомогенный катализатор фталоцианина кобальта, который позволяет восстановить щелочной раствор путем окисления меркаптидов до дисульфидов. Отработанный воздух сепарируется из отстойника дисульфидов, а дисульфидное масло отправляется на нужды потребителю, регенерированная щелочь с малым содержанием меркаптидов возвращается в экстрактор.

\begin{figure}
	\centering
	\includegraphics[width=1\linewidth]{images/Eth}
	\caption{Схема двухсекционной демеркаптанизации компании Merichem (THIOLEX).}
	\label{fig:THIOLEX}
\end{figure}

\begin{figure}
	\centering
	\includegraphics[width=0.6\linewidth]{images/fibr}
	\caption{Схема работы фиброволокнистого контактного устройства компании Merichem.}
	\label{fig:FIBERFILM}
\end{figure}

Так же стоит отметить предложение от компании Merichem рисунок \cref{fig:THIOLEX}, 
процесс THIOLEX, демеркаптанизацию проводят в двухсекционном отстойнике, где щелочная раствор смачивает фиброволокнистое контактное устройство рисунок \cref{fig:FIBERFILM}, затем СУГ стекает по смоченному щелочью контактному устройству и на поверхности раздела фаз реагирует с ней, $H_2S$ и меркаптаны переходят щелочную фазу в виде сульфидов и меркаптидов натрия. Обе фазы отделяются друг от друга в двухсекционном отстойнике, насыщенный сульфидами и меркаптидами натрия щелочной раствор выводится с нижней части отстойника и под собственным давлением направляется на регенерацию, а очищенный до \num{10} ppm c первой секции СУГ поступает на доочистку во вторую секцию отстойника и очищается до \num{5} ppm, так же отмечено что происходит унос щелочного раствора до \num{0,1} ppm. Регенерацию проводят в колонне, где насыщенный щелочной раствор с гомогенным катализатором поднимаясь верх по колонне контактирует с воздухом и происходит окисление содержащихся в щелочном растворе сульфидами и меркаптидами натрия до дисульфидов. Для снижения плотности дисульфидного масла, с целью облегчения его отделения от щелочного раствора в отстойнике дисульфидов, в отстойник подается расчетное количество нафты. Дисульфидное масло с более низким значением плотности, выводится из колонны регенератора для отгрузки на потребительские нужды.

%------------------------------------------------------------------------------------------
\subsection{Влияние контактных устройств на гидродинамику и массообмен процесса.} \label{sec:ch1/sec3}

Схема противоточной системы хемосорбцонной экстракции из двух (частично) растворимых друг в друге жидкостей СУГ и Водно-щелочной раствор представлена на рисунке \cref{fig:cheme}.

\begin{figure}
	\centering
	\includegraphics[width=0.6\linewidth]{images/fibr}
	\caption{Заглушка.}
	\label{fig:cheme}
\end{figure}

Взаимодействие фаз при демеркаптанизации может быть ступенчатым и непрерывным. Ступенчатая проводится проводится в смесительно-отстойных аппаратах \cref{fig:THIOLEX}, а непрерывная – в гравитационных и центробежных аппаратах \cref{fig:THIOLEX}. В гравитационных экстракторах противоточное движение фаз осуществляется под действием поля силы тяжести за счет разности плотностей фаз.

Гравитационные экстракторы характеризуются невысокой эффективностью из-за малой поверхности контакта фаз, вызванной большими размерами капель. Кроме этого, в аппаратах размерами более $1,5$ метров возникают значительные поперечные неравномерности (масштабные эффекты), что снижает эффективность разделения смеси.

Одним из распространенных аппаратов в промышленности демеркаптанизации являются колонны с ситчатыми тарелками.  В них водно-щелочная фаза сплошная занимает весь объем аппарата и перетекает с тарелки на тарелку, другая дисперсная фаза СУГ диспергируется сквозь отверстия тарелки, движется в межтарельчатом пространстве и,  достигая следующей тарелки, коалесцирует. В результате многократного диспергирования и коалесценции интенсивность процесса демеркаптанизации возрастает. Наличие ситчатых тарелок способствует снижению обратного перемешивания. Находят применение и аппараты с насадкой, аппараты с пульсационные и вибрационные аппараты.
 


